% platex + dvipdfmx で最低3回コンパイルが必要
\documentclass{jsarticle}
\let\ifdraft\relax % 後で movie15 パッケージが ifdraft.sty が読み込むことへの対策
\usepackage[dvipdfmx]{hyperref} % \movieref を使う場合に必要
\usepackage[dvipdfmx]{movie15_dvipdfmx}
\西暦
\title{動画のmovie15\_dvipdfmxによる取り込み}
\author{Acetaminophen}
\begin{document}
\maketitle

元の動画は滋賀大学の熊澤さんがさまざまな\TeX パッケージのサンプルを公開しているサイト
\url{http://www.biwako.shiga-u.ac.jp/sensei/kumazawa/tex/movie15.html}
で配布されている「彦根城外堀」の mp4 ファイルである。

\begin{figure}[ht]
\centering
\includemovie[label=hikone]{80mm}{60mm}{movie1501.mp4} \\
\begin{tabular}{|c|c|c|}
\movieref{hikone}{再生}
 & \movieref[pause]{hikone}{一時停止/再生}
 & \movieref[stop]{hikone}{停止}
\end{tabular}
\end{figure}

\verb+\includemovie+のオプションで\verb+poster=hoge.png+と指定すれば、
動画や音声を再生していない時に表示する画像を\verb+hoge.png+に指定できる。

\end{document}