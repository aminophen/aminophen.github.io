\documentclass{jsarticle}
\usepackage{noconflict}
\usepackage{testingconflictA}
\prefix{A}
\save*{testingconflictX,testingconflictY,testingconflictZ}
\usepackage{testingconflictB}
\prefix{B}
\save*{testingconflictX,testingconflictY,testingconflictZ}
\usepackage{testingconflictC}
\prefix{C}
\save*{testingconflictX,testingconflictY,testingconflictZ}
\title{\textsf{noconflict} パッケージを使ってみる4}
\author{アセトアミノフェン}
\begin{document}
\maketitle

\testingconflictX{test}
\testingconflictY{hoge}
\testingconflictZ{fuga}

\AtestingconflictX{てすと}
\AtestingconflictY{ほげ}
\AtestingconflictZ{ふが}

\BtestingconflictX{テスト}
\BtestingconflictY{ホゲ}
\BtestingconflictZ{フガ}

\CtestingconflictX{foo}
\CtestingconflictY{bar}
\CtestingconflictZ{baz}

ここまでは
\begin{itemize}
\item 元のコマンド名:空
\item Aが付いたコマンド名:パッケージAのもの
\item Bが付いたコマンド名:パッケージBのもの
\item Cが付いたコマンド名:パッケージCのもの
\end{itemize}
となっている。しかし、ここでBのマクロをもとのコマンド名に復帰させてみる。

\restore{B}

\testingconflictX{test}
\testingconflictY{hoge}
\testingconflictZ{fuga}

\AtestingconflictX{てすと}
\AtestingconflictY{ほげ}
\AtestingconflictZ{ふが}

\BtestingconflictX{テスト}
\BtestingconflictY{ホゲ}
\BtestingconflictZ{フガ}

\CtestingconflictX{foo}
\CtestingconflictY{bar}
\CtestingconflictZ{baz}

このように、元のコマンド名からの出力が先ほどまでの空の状態ではなくBのマクロに変化した。
この場合、新たに定義したBが付いたコマンドも失われない。

\end{document}