\documentclass{jsarticle}
\usepackage{noconflict}
\usepackage{testingconflictA}
\save{A}{testingconflictX,testingconflictY,testingconflictZ}
\usepackage{testingconflictB}
\save{B}{testingconflictX,testingconflictY,testingconflictZ}
\usepackage{testingconflictC}
\save{C}{testingconflictX,testingconflictY,testingconflictZ}
\title{\textsf{noconflict} パッケージを使ってみる5}
\author{アセトアミノフェン}
\begin{document}
\maketitle

\testingconflictX{test}
\testingconflictY{hoge}
\testingconflictZ{fuga}

\AtestingconflictX{てすと}
\AtestingconflictY{ほげ}
\AtestingconflictZ{ふが}

\BtestingconflictX{テスト}
\BtestingconflictY{ホゲ}
\BtestingconflictZ{フガ}

\CtestingconflictX{foo}
\CtestingconflictY{bar}
\CtestingconflictZ{baz}

ここで
\begin{itemize}
\item コマンドXを3つも使い分ける必要はなく、パッケージAのマクロだけで良い
\item 同様にYはパッケージBのもの、ZはパッケージCのものだけで良い
\end{itemize}
という事情から、いちいち接頭辞付にしたくない場合があるかもしれない。そうした場合はマクロのリネームを行うと、以後は元のコマンド名で

\rename{AtestingconflictX}{testingconflictX}
\rename{BtestingconflictY}{testingconflictY}
\rename{CtestingconflictZ}{testingconflictZ}

\testingconflictX{test}
\testingconflictY{hoge}
\testingconflictZ{fuga}

このように、任意のパッケージの好きなコマンドを自由に組み合わせて利用できるようになる。なお、このときリネームされた接頭辞付コマンドは以下のように空になる。

\AtestingconflictX{てすと}
\AtestingconflictY{ほげ}
\AtestingconflictZ{ふが}

\BtestingconflictX{テスト}
\BtestingconflictY{ホゲ}
\BtestingconflictZ{フガ}

\CtestingconflictX{foo}
\CtestingconflictY{bar}
\CtestingconflictZ{baz}

\end{document}