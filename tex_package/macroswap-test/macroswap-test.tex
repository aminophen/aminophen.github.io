\documentclass{jsarticle}
\usepackage{macroswap}
\newcommand{\myfirst}{First}
\newcommand{\myend}{End}
\title{\textsf{macroswap} パッケージを使ってみる}
\author{アセトアミノフェン}
\begin{document}
\maketitle

簡単な例として、以下のように定義します:
\begin{verbatim}
\newcommand{\myfirst}{First}
\newcommand{\myend}{End}
\end{verbatim}
すなわち \verb|\myfirst| で First が出力され、\verb|\myend| で End が出力されます。

試しに一度出力してみます:\myfirst \myend
(ここでは FirstEnd と出ているはずです)


ここで \textsf{macroswap} パッケージの \verb|\macroswap| や \verb|\gmacroswap| を使います。
このコマンドは引数を2つとり、それぞれの引数にはマクロ名を指定します。
\verb|\macroswap| や \verb|\gmacroswap| が記述されると、以降で2つのマクロの名称が逆転します。
\begin{itemize}
\item \verb|\macroswap| は \verb|\begingroup| と \verb|\endgroup| の間に書かれていれば、そのグループ外では無効
\item \verb|\gmacroswap| は以降のマクロ名をグローバルに変更
\end{itemize}
という違いがあります。


まずは \verb|\macroswap| を試します:
\begingroup
\macroswap{myfirst}{myend}
\myfirst \myend
(ここでは EndFirst と出ているはずです)
\endgroup

グループの外では FirstEnd に戻っているはずです:
\myfirst \myend


次に \verb|\gmacroswap| を試します:
\begingroup
\gmacroswap{myfirst}{myend}
\myfirst \myend
(ここでは EndFirst と出ているはずです)
\endgroup

グループの外でも逆転した EndFirst のままになっているはずです:
\myfirst \myend

\end{document}