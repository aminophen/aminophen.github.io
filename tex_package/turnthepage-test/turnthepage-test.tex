\documentclass{jsarticle}
\usepackage[english]{turnthepage}
\renewcommand{\turnthepage}{ページをめくりなさい。}
\title{\textsf{turnthepage} パッケージを使ってみる}
\author{アセトアミノフェン}
\begin{document}
\maketitle

CTAN の \textsf{turnthepage} パッケージの利用例です。

機能は、見開きで右にあたる奇数ページ下部(最後のページを除く)に
\begin{quote}
\textit{Turn the page.}
\end{quote}
というメッセージを表示するだけです。

ただし、英語・フランス語・オランダ語・ドイツ語しかサポートしていないため、この例では日本語化のため
\begin{verbatim}
\renewcommand{\turnthepage}{ページをめくりなさい。}
\end{verbatim}
という再定義を与えています。

\clearpage

次のページになりました。

このようなパッケージは、例えば
\begin{itemize}
\item 試験問題で各ページに空白部分を設けた場合に、受験者に次のページにも問題があることを教える
\end{itemize}
という目的で用いることができます。

\clearpage

例えば、ここに問題文があるとします。

\clearpage

先ほどのページに余白が多いため、「ページをめくりなさい。」というメッセージがなければこのページに問題文があることに気づきにくくなってしまいます。

\clearpage

最後のページになりました。ここにはメッセージが出ないはずです。

注意:タイプセットは最低2回必要です。1回だけではページ数を取得できていないため、メッセージを出力すべきページを判断できていないためだと思われます。

\end{document}